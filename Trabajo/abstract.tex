\chapter*{Resumen}
El presente trabajo describe el concepto de confinamiento magnético, en particular en el reactor tipo \textit{stellarator}, y analiza si la existencia de islas magnéticas o turbulencia es favorable o adversa
al confinamiento del plasma en el reactor.\par
En primer lugar, se presenta una introducción a la energía nuclear, particularizando en la de fusión, continúa
un análisis al concepto de confinamiento y su duración, así como métodos que lo aumentan o fenómenos físicos que lo reducen y termina con medidas
de interés dentro del plasma y sus aplicaciones en la mejora del confinamiento.\par
En segundo lugar, se toma contacto en profundidad con los reactores tipo \textit{stellarator}, pasando por su historia, 
conceptos fundamentales y terminar con el \textit{stellarator} español en el CIEMAT: TJ-II.\par
En última instancia, se realiza un análisis más exhaustivo del concepto de turbulencia con el apoyo
de herramientas matemáticas como el campo eléctrico, finalizando con la fijación del concepto de isla
magnética y su medición.\par
\paragraph{Palabras clave:} plasma, \textit{stellarator}, confinamiento, magnético
\chapter*{Abstract}
This paper describes the concept of magnetic confinement, in the stellarator type reactor, and analyzes whether the existence of magnetic islands or turbulence is favorable or adverse
to the confinement of the plasma in the reactor.\par
First, there is an introduction to nuclear energy, particularly in the field of fusion, which continues 
an analysis of the concept of confinement and its duration, as well as methods that increase it or physical 
phenomena that reduce it and end up with measures of interest within plasma and its applications in improving confinement.\par
Secondly, the stellarator-type reactors are contacted in depth, passing through their history, 
fundamental concepts and ending with the Spanish stellarator at CIEMAT: TJ-II.\par
Ultimately, a more comprehensive analysis of the concept of turbulence is carried out with the support
of mathematical tools such as the electric field, ending with the establishment of the island concept
and its measurement.\par
\paragraph{Keywords:} plasma, stellarator, confinement, magnetic