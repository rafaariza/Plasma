\chapter{Introducción}\label{ch:intro}
\section{Energía}
La demanda global energética está aumentando en los últimos años principalmente debido al crecimiento demográfico y económico.
El continuo desarrollo de la sociedad moderna requiere que las fuentes de energía sean sostenibles
y respetuosas con el medio ambiente. Sin embargo, hoy en día más del 80\% de la energía mundial proviene
de combustibles fósiles\cite{/content/publication/caf32f3b-en}, incluyendo carbón, petróleo y gas natural, que están limitados en reserva.
Además, las emisiones de CO$_2$ de este tipo de energía es la principal contribución al aumento del 
efecto invernadero y tiene un efecto importante en el cambio climático. Se
sabe que para nuestro entorno de vida el nivel asequible de aumento de la
temperatura media por encima de los niveles antes de la era industrial es de 2 \centigrade C, más allá del cual es irreversible y casi
se espera que ocurra un cambio climático catastrófico e incontrolable. Este aumento ya ha alcanzado los 0.78 \centigrade C\cite{ip02000c}.
Por lo tanto, es urgente encontrar una manera de reducir la energía relacionada con
los combustibles fósiles. Un alivio a corto plazo sería el desarrollo de nuevas tecnologías para
reducir las emisiones de CO$_2$ de las plantas de energía fósil y mejorar el almacenamiento de CO$_2$ a gran escala,
mientras que una solución a largo plazo debería considerar las alternativas a los combustibles fósiles. Los candidatos con recursos suficientes para hacerse cargo del abastecimiento son la energía solar, la fisión nuclear y
la fusión nuclear\cite{Freidberg:1186225}.\par
La energía solar es teóricamente amplia e inagotable, pero su intermitencia (luz solo en días sin nubes) y la baja densidad energética (se requiere una gran superficie) hacen difícil construir
una planta de energía solar para producir una cantidad significativa de energía base. La fisión nuclear es
una fuente de energía bien establecida y ha estado produciendo electricidad de carga base durante décadas.
Sin embargo, la eliminación de los residuos nucleares de larga y media vida junto al riesgo de accidente debido a la
reacción en cadena intrínseca de la fisión nuclear ha sido durante mucho tiempo una preocupación pública.\par
La fusión nuclear promete una solución limpia y segura para nuestras necesidades energéticas a largo plazo\cite{Freidberg:1186225}. 
Primero, las reservas de combustible son abundantes. Para la reacción se necesitan deuterio y tritio, el deuterio puede ser 
extraído del agua de mar; el tritio no se produce de forma natural, pero puede ser obtenido a partir del isótopo de lito $^6$Li. 
Segundo, las reacciones de fusión nuclear no emiten gases de efecto invernadero o cualquier otro daño químico a la atmósfera. 
Tercero, la fusión nuclear es intrínsecamente segura. El combustible de fusión se introduce continuamente en el reactor a una velocidad que sostiene la reacción durante sólo unas pocas decenas
de segundos en cada instante. La reacción de fusión sólo puede ocurrir bajo una temperatura muy alta
y un campo de confinamiento suficientemente preciso y sin reacción en cadena. Cualquier manipulación incorrecta detendrá la reacción.